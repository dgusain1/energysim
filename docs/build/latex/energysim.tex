%% Generated by Sphinx.
\def\sphinxdocclass{report}
\documentclass[letterpaper,10pt,english]{sphinxmanual}
\ifdefined\pdfpxdimen
   \let\sphinxpxdimen\pdfpxdimen\else\newdimen\sphinxpxdimen
\fi \sphinxpxdimen=.75bp\relax

\PassOptionsToPackage{warn}{textcomp}
\usepackage[utf8]{inputenc}
\ifdefined\DeclareUnicodeCharacter
% support both utf8 and utf8x syntaxes
  \ifdefined\DeclareUnicodeCharacterAsOptional
    \def\sphinxDUC#1{\DeclareUnicodeCharacter{"#1}}
  \else
    \let\sphinxDUC\DeclareUnicodeCharacter
  \fi
  \sphinxDUC{00A0}{\nobreakspace}
  \sphinxDUC{2500}{\sphinxunichar{2500}}
  \sphinxDUC{2502}{\sphinxunichar{2502}}
  \sphinxDUC{2514}{\sphinxunichar{2514}}
  \sphinxDUC{251C}{\sphinxunichar{251C}}
  \sphinxDUC{2572}{\textbackslash}
\fi
\usepackage{cmap}
\usepackage[T1]{fontenc}
\usepackage{amsmath,amssymb,amstext}
\usepackage{babel}



\usepackage{times}
\expandafter\ifx\csname T@LGR\endcsname\relax
\else
% LGR was declared as font encoding
  \substitutefont{LGR}{\rmdefault}{cmr}
  \substitutefont{LGR}{\sfdefault}{cmss}
  \substitutefont{LGR}{\ttdefault}{cmtt}
\fi
\expandafter\ifx\csname T@X2\endcsname\relax
  \expandafter\ifx\csname T@T2A\endcsname\relax
  \else
  % T2A was declared as font encoding
    \substitutefont{T2A}{\rmdefault}{cmr}
    \substitutefont{T2A}{\sfdefault}{cmss}
    \substitutefont{T2A}{\ttdefault}{cmtt}
  \fi
\else
% X2 was declared as font encoding
  \substitutefont{X2}{\rmdefault}{cmr}
  \substitutefont{X2}{\sfdefault}{cmss}
  \substitutefont{X2}{\ttdefault}{cmtt}
\fi


\usepackage[Bjarne]{fncychap}
\usepackage{sphinx}

\fvset{fontsize=\small}
\usepackage{geometry}


% Include hyperref last.
\usepackage{hyperref}
% Fix anchor placement for figures with captions.
\usepackage{hypcap}% it must be loaded after hyperref.
% Set up styles of URL: it should be placed after hyperref.
\urlstyle{same}

\addto\captionsenglish{\renewcommand{\contentsname}{Contents:}}

\usepackage{sphinxmessages}
\setcounter{tocdepth}{1}



\title{energysim}
\date{Jul 23, 2020}
\release{2.0}
\author{Digvijay Gusain}
\newcommand{\sphinxlogo}{\vbox{}}
\renewcommand{\releasename}{Release}
\makeindex
\begin{document}

\pagestyle{empty}
\sphinxmaketitle
\pagestyle{plain}
\sphinxtableofcontents
\pagestyle{normal}
\phantomsection\label{\detokenize{index::doc}}


\sphinxstylestrong{energysim} is a python based cosimulation tool designed to simplify multi\sphinxhyphen{}energy cosimulations. The tool was initially called \sphinxstylestrong{FMUWorld}, sincei t focussed exclusively on combining models developed and packaged as Functional Mockup Units (FMUs). However, it has since been majorly updated to become a more generalisable cosimulation tool to include a more variety of energy system simulators.

The idea behind development of \sphinxstylestrong{energysim} is to simplify cosimulation to focus on the high\sphinxhyphen{}level applications, such as energy system planning, evaluation of control strategies, etc., rather than low\sphinxhyphen{}level cosimulation tasks such as message exchange, time progression coordination, etc.


\chapter{Initialization}
\label{\detokenize{index:initialization}}
\sphinxcode{\sphinxupquote{world}} is your cosimulation canvas. It creates a \sphinxcode{\sphinxupquote{world}} object on which simulators can be freely added. \sphinxcode{\sphinxupquote{world}} accepts the following parameters :
\begin{quote}
\begin{itemize}
\item {} 
\sphinxcode{\sphinxupquote{start\_time}} : (0 by default).

\item {} 
\sphinxcode{\sphinxupquote{stop\_time}} : (1000 by default).

\item {} 
\sphinxcode{\sphinxupquote{logging}} : (False by default). Gives users update on simulation progress.

\item {} 
\sphinxcode{\sphinxupquote{exchange}} : (60 by default). Users can specify instance of information exchange.

\end{itemize}

\begin{sphinxVerbatim}[commandchars=\\\{\}]
\PYG{g+gp}{\PYGZgt{}\PYGZgt{}\PYGZgt{} }\PYG{n}{my\PYGZus{}world} \PYG{o}{=} \PYG{n}{world}\PYG{p}{(}\PYG{n}{start\PYGZus{}time}\PYG{o}{=}\PYG{l+m+mi}{0}\PYG{p}{,} \PYG{n}{stop\PYGZus{}time}\PYG{o}{=}\PYG{l+m+mi}{1000}\PYG{p}{,} \PYG{n}{logging}\PYG{o}{=}\PYG{k+kc}{True}\PYG{p}{,} \PYG{n}{exchange}\PYG{o}{=}\PYG{l+m+mi}{60}\PYG{p}{)}
\end{sphinxVerbatim}
\end{quote}


\chapter{Specifying Simulators}
\label{\detokenize{index:specifying-simulators}}
After initializing the world cosimulation object, you can add simulators
to the world by \sphinxcode{\sphinxupquote{add\_simulator()}}. This is done by specifying
the following :
\begin{quote}
\begin{itemize}
\item {} 
\sphinxcode{\sphinxupquote{sim\_type}} : ‘fmu’, ‘powerflow’, ‘csv’, ‘signal’

\item {} 
\sphinxcode{\sphinxupquote{sim\_name}} : It is essential that you assign the simulator a unique name.

\item {} 
\sphinxcode{\sphinxupquote{sim\_loc}} : Use a raw string address of simulator location.

\item {} 
\sphinxcode{\sphinxupquote{outputs}} : specify the outputs that need to be recorded during simulation from the simulator.

\item {} 
\sphinxcode{\sphinxupquote{inputs}} : specify the inputs of the simulator.

\item {} 
\sphinxcode{\sphinxupquote{step\_size}} : step size for simulator (1e\sphinxhyphen{}3 by default).

\end{itemize}

\begin{sphinxVerbatim}[commandchars=\\\{\}]
\PYG{g+gp}{\PYGZgt{}\PYGZgt{}\PYGZgt{} }\PYG{n}{my\PYGZus{}world}\PYG{o}{.}\PYG{n}{add\PYGZus{}simulator}\PYG{p}{(}\PYG{n}{sim\PYGZus{}type}\PYG{o}{=}\PYG{l+s+s1}{\PYGZsq{}}\PYG{l+s+s1}{fmu}\PYG{l+s+s1}{\PYGZsq{}}\PYG{p}{,} \PYG{n}{sim\PYGZus{}name}\PYG{o}{=}\PYG{l+s+s1}{\PYGZsq{}}\PYG{l+s+s1}{FMU1}\PYG{l+s+s1}{\PYGZsq{}}\PYG{p}{,}
\PYG{g+gp}{\PYGZgt{}\PYGZgt{}\PYGZgt{} }\PYG{n}{sim\PYGZus{}loc}\PYG{o}{=}\PYG{o}{/}\PYG{n}{path}\PYG{o}{/}\PYG{n}{to}\PYG{o}{/}\PYG{n}{sim}\PYG{p}{,} \PYG{n}{inputs}\PYG{o}{=}\PYG{p}{[}\PYG{p}{]}\PYG{p}{,} \PYG{n}{outputs}\PYG{o}{=}\PYG{p}{[}\PYG{l+s+s1}{\PYGZsq{}}\PYG{l+s+s1}{var1}\PYG{l+s+s1}{\PYGZsq{}}\PYG{p}{,}\PYG{l+s+s1}{\PYGZsq{}}\PYG{l+s+s1}{var2}\PYG{l+s+s1}{\PYGZsq{}}\PYG{p}{]}\PYG{p}{,} \PYG{n}{step\PYGZus{}size}\PYG{o}{=}\PYG{l+m+mi}{1}\PYG{p}{)}
\end{sphinxVerbatim}
\end{quote}


\chapter{Connections between simulators}
\label{\detokenize{index:connections-between-simulators}}
The connections between simulators can be specified with a dictionary
as follows :

\begin{sphinxVerbatim}[commandchars=\\\{\}]
\PYG{g+gp}{\PYGZgt{}\PYGZgt{}\PYGZgt{} }\PYG{n}{connections} \PYG{o}{=} \PYG{p}{\PYGZob{}}\PYG{l+s+s1}{\PYGZsq{}}\PYG{l+s+s1}{sim1.output\PYGZus{}variable1}\PYG{l+s+s1}{\PYGZsq{}} \PYG{p}{:} \PYG{l+s+s1}{\PYGZsq{}}\PYG{l+s+s1}{sim2.input\PYGZus{}variable1}\PYG{l+s+s1}{\PYGZsq{}}\PYG{p}{,}
\PYG{g+gp}{\PYGZgt{}\PYGZgt{}\PYGZgt{} }   \PYG{l+s+s1}{\PYGZsq{}}\PYG{l+s+s1}{sim3.output\PYGZus{}variable2}\PYG{l+s+s1}{\PYGZsq{}} \PYG{p}{:} \PYG{l+s+s1}{\PYGZsq{}}\PYG{l+s+s1}{sim4.input\PYGZus{}variable2}\PYG{l+s+s1}{\PYGZsq{}}\PYG{p}{,}
\PYG{g+gp}{\PYGZgt{}\PYGZgt{}\PYGZgt{} }   \PYG{l+s+s1}{\PYGZsq{}}\PYG{l+s+s1}{sim1.output\PYGZus{}variable3}\PYG{l+s+s1}{\PYGZsq{}} \PYG{p}{:} \PYG{l+s+s1}{\PYGZsq{}}\PYG{l+s+s1}{sim2.input\PYGZus{}variable3}\PYG{l+s+s1}{\PYGZsq{}}\PYG{p}{,}\PYG{p}{\PYGZcb{}}
\PYG{g+gp}{\PYGZgt{}\PYGZgt{}\PYGZgt{} }\PYG{n}{my\PYGZus{}world}\PYG{o}{.}\PYG{n}{add\PYGZus{}connections}\PYG{p}{(}\PYG{n}{connections}\PYG{p}{)}
\end{sphinxVerbatim}

This dictionary can be passed onto the world object.


\chapter{Initializing simulator variables}
\label{\detokenize{index:initializing-simulator-variables}}
To provide initial values to the simulators, the initial values can be
specified by providing a \sphinxcode{\sphinxupquote{init}} dict :

\begin{sphinxVerbatim}[commandchars=\\\{\}]
\PYG{g+gp}{\PYGZgt{}\PYGZgt{}\PYGZgt{} }\PYG{n}{initializations} \PYG{o}{=} \PYG{p}{\PYGZob{}}\PYG{l+s+s1}{\PYGZsq{}}\PYG{l+s+s1}{sim\PYGZus{}name1}\PYG{l+s+s1}{\PYGZsq{}} \PYG{p}{:} \PYG{p}{(}\PYG{p}{[}\PYG{l+s+s1}{\PYGZsq{}}\PYG{l+s+s1}{sim\PYGZus{}variables}\PYG{l+s+s1}{\PYGZsq{}}\PYG{p}{]}\PYG{p}{,} \PYG{p}{[}\PYG{n}{values}\PYG{p}{]}\PYG{p}{)}\PYG{p}{,}
\PYG{g+gp}{\PYGZgt{}\PYGZgt{}\PYGZgt{} }                   \PYG{l+s+s1}{\PYGZsq{}}\PYG{l+s+s1}{sim\PYGZus{}name2}\PYG{l+s+s1}{\PYGZsq{}} \PYG{p}{:} \PYG{p}{(}\PYG{p}{[}\PYG{l+s+s1}{\PYGZsq{}}\PYG{l+s+s1}{sim\PYGZus{}variables}\PYG{l+s+s1}{\PYGZsq{}}\PYG{p}{]}\PYG{p}{,} \PYG{p}{[}\PYG{n}{values}\PYG{p}{]}\PYG{p}{)}\PYG{p}{\PYGZcb{}}
\PYG{g+gp}{\PYGZgt{}\PYGZgt{}\PYGZgt{} }\PYG{n}{options} \PYG{o}{=} \PYG{p}{\PYGZob{}}\PYG{l+s+s1}{\PYGZsq{}}\PYG{l+s+s1}{init}\PYG{l+s+s1}{\PYGZsq{}} \PYG{p}{:} \PYG{n}{initializations}\PYG{p}{\PYGZcb{}}
\PYG{g+gp}{\PYGZgt{}\PYGZgt{}\PYGZgt{} }\PYG{n}{my\PYGZus{}world}\PYG{o}{.}\PYG{n}{options}\PYG{p}{(}\PYG{n}{options}\PYG{p}{)}
\end{sphinxVerbatim}


\chapter{Simulate}
\label{\detokenize{index:simulate}}
Finally, the \sphinxcode{\sphinxupquote{simulate()}} function can be called to simulate the world.
This returns a dictionary with simulator name as keys and the results of
the simulator as pandas dataframe.
\sphinxcode{\sphinxupquote{pbar}} can be used to toggle the progress bar for the simulation.

\begin{sphinxVerbatim}[commandchars=\\\{\}]
\PYG{g+gp}{\PYGZgt{}\PYGZgt{}\PYGZgt{} }\PYG{n}{my\PYGZus{}world}\PYG{o}{.}\PYG{n}{simulate}\PYG{p}{(}\PYG{n}{pbar}\PYG{o}{=}\PYG{k+kc}{True}\PYG{p}{)}
\end{sphinxVerbatim}


\chapter{Package Info}
\label{\detokenize{index:package-info}}
Author : Digvijay Gusain

Email : \sphinxhref{mailto:digvijay.gusain29@gmail.com}{digvijay.gusain29@gmail.com}

Version : 2.0


\chapter{Indices and tables}
\label{\detokenize{index:indices-and-tables}}\begin{itemize}
\item {} 
\DUrole{xref,std,std-ref}{genindex}

\item {} 
\DUrole{xref,std,std-ref}{modindex}

\item {} 
\DUrole{xref,std,std-ref}{search}

\end{itemize}



\renewcommand{\indexname}{Index}
\printindex
\end{document}